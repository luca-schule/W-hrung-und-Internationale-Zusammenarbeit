\documentclass[10pt,a4paper, ngerman]{beamer}

\include{beamer}

\AtBeginSection{\frame{\frametitle{Gliederung}\tableofcontents[currentsection]}}

\setbeamercovered{transparent}
\author{Luca Kiebel \and Paul Hermann}
\title{Währungen und Internationale Zusammenarbeit}
%\subtitle{subtitle}
\date{\today}
\institute[HBBK]{Hans-Böckler-Berufskolleg}
\setlength{\itemsep}{10pt}
\begin{document}
\begin{frame}
\titlepage
\end{frame}

\section{Begriffsklärung}
\subsection{Was ist Währung?}
\begin{frame}{Was ist Währung?}{Begriffsklärung}
\pftn{http://bit.ly/2EShb0r{,} "Währung und internationale Zusammenarbeit"{,} Seite 226}
\begin{itemize}
\item Verfassung und Ordnung des Geldwesens eines Staates \pause
\item Allgeimer Sprachgebrauch: Geldeinheit eines Staates \pause
\item  Werden durch Abkürzungen oder Währungssymbole dargestellt 
\end{itemize}
\end{frame}

\subsection{Was sind Devisen?}
\begin{frame}{Was sind Devisen?}{Begriffsklärung}
\begin{itemize}
		\item Zahlungsmittel in fremder Währung\ftn{1}{https://www.duden.de/rechtschreibung/Devise} \pause
		\item Meist in Form von Guthaben bei ausländischen Banken\ftn{2}{https://de.wikipedia.org/wiki/Devisen}
\end{itemize}
\end{frame}

\subsection{Was ist der Wechselkurs?}
\begin{frame}{Was ist der Wechselkurs?}{Begriffsklärung}
\pftn{http://bit.ly/2EShb0r{,} "Währung und internationale Zusammenarbeit"{,} Seite 226}
\begin{itemize}
\item Das Austauschverhältnis zwischen zwei Währungen \pause
\item Richtet sich nach Angebot und Nachfrage
\end{itemize}
\end{frame}

\subsection{Was sind Primär- und Sekundäreinkommen?}
\begin{frame}{Was sind Primär- und Sekundäreinkommen?}
\begin{itemize}
\item Das Primäreinkommen ergiebt sich aus dem Marktprozess.\ftn{1}{http://wirtschaftslexikon.gabler.de/Definition/primaereinkommen.html} \pause
\item Sekundäreinkommen \ftn{2}{http://bit.ly/2EShb0r{,} "Währung und internationale Zusammenarbeit"{,} Seite 241}
\end{itemize}
\end{frame}

\section{Wechselkurse}
\subsection{Feste Wechselkurse}
\begin{frame}{Feste Wechselkurse}{Wechselkurse}
\begin{itemize}
\item Interventionen am Devisenmarkt halten Feste Wechselkurse aufrecht \pause
\item Angebot wird Nachfrage angepasst \pause
\item => Die Zentralbanken müssen eigene Geldpolitik auf Geldstabilität ausrichten
\end{itemize}
\pftn{http://bit.ly/2EShb0r{,} "Währung und internationale Zusammenarbeit"{,} Seite 230}
\end{frame}

\subsection{Flexible Wechselkurse}
\begin{frame}
\pftn{http://bit.ly/2EShb0r{,} "Währung und internationale Zusammenarbeit"{,} Seite 231}
\begin{itemize}
\item Die meisten Währungen haben heute flexible Wechselkurse \pause
\item Angebot und Nachfrage \pause
\item Exporteure können mit Derivaten Wechselkursrisiken eingrenzen
\end{itemize}
\end{frame}

\section{Europäisches Währungssystem}
\subsection{von '79 bis '98}
\begin{frame}{Europäisches Währungssystem von  '79 bis '98}
\pftn{http://bit.ly/2EShb0r{,} "Währung und internationale Zusammenarbeit"{,} Seite 234}
\begin{itemize}
\item Vereinbarung für Leitkurse mit geringen Schwankungen \pause
\item Zentralbanken verpflichtet zu intervenieren
\end{itemize}
\end{frame}

\subsection{ab '98}
\begin{frame}{Europäisches Währungssystem ab '98}
\pftn{https://de.wikipedia.org/wiki/Wechselkursmechanismus\_II}
\begin{itemize}
\item Stabilisierung des gemeinsamen Binnenmarktes \pause
\item Schwankungsgrenzen von 15 \%
\end{itemize}
\end{frame}

\section{Zahlungsbilanz}
\begin{frame}{Zahlungsbilanz}
\begin{figure}
\centering
\includegraphics[height=0.9\textheight]{zahlungsbilanz_de}
\pftn{http://bit.ly/2EShb0r{,} "Währung und internationale Zusammenarbeit"{,} Seite 237}
\end{figure}

\end{frame}

\end{document}